\chapter{Introducción}
\label{capitulo1}

Los avances recientes en la informática han visto a los lenguajes funcionales conducir a una mejor productividad en muchas industrias, y la industria de los videojuegos y medios interactivos también se pueda beneficiar de estos avances con el uso de lenguajes funcionales. Los lenguajes de programación funcionales ofrecen muchas ventajas comparadas con los lenguajes imperativos que son ampliamente utilizados en esta industria. Los lenguajes funcionales son mucho más concisos en comparación con los lenguajes imperativos, y está probado, que en ciertas ocasiones, los lenguajes funcionales muestran un mejor rendimiento que sus contrapartes imperativas. También permiten el uso de poderosas abstracciones que pueden ser utilizadas para mejorar la estructura y modularidad del código. Los lenguajes funcionales también permiten el polimorfismo que promueve la reutilización del código y menos redundancia en los programas.

Es por estas razones que este proyecto se propone la creación de un framework que permita la construcción de juegos en el lenguaje \emph{Haskell}, a través de una interfaz que permita, de manera transparente al programador, implementar la funcionalidad requerida para su propio proyecto sin tener que preocuparse por la lógica y funcionalidad subyacente a todo videojuego, que en el lenguaje \emph{Haskell} presenta sus propios retos y complicaciones. Algunas aplicaciones que esta herramienta puede presentar a diversos campos son:

\begin{itemize}
\item Industria: un entorno funcional ayudaría a reducir los errores en el código final y reducir el tiempo de producción.
\item Entretenimiento: esta herramienta podría ofrecer un avance en la calidad y cantidad de la producción de medios interactivos.
\item Educación: esta herramienta puede ser usada para la visualización y modelado de simulaciones, debido a su similitud con los videojuegos.
\end{itemize}

Conceptualmente, alejarse de un lenguaje imperativo (Java, C\#, C++, etc.) nos ayudará a mejorar la capacidad de comprensión del código, ya que la Programación funcional se centra en el problema en sí. Es decir, la programación funcional se centra en qué hacer en lugar de cómo hacerlo. Proporcionando una mejor abstracción para la resolución de problemas. Eso puede sonar como un pequeño beneficio por el esfuerzo que implica cambiar la forma en que abordamos el código. Pero cuando nos damos cuenta de que tener una mejor comprensión del código nos lleva a una gran mejora en la depuración y mantenimiento, podemos ver claramente cómo se traduce en ahorros de tiempo y dinero que afectarán el éxito general del proyecto. Hoy en día, varias startups dependen de cuán rápido se puede llegar a una solución de trabajo, qué tan fácil es escalar desde allí y cómo se pueden hacer esas cosas con la menor cantidad de dinero posible.

A través del paradigma de Programación Funcional ganamos elegancia y simplicidad, descomposición más fácil de los problemas y código más estrechamente relacionado con el dominio del problema. Esto también nos conduce a pruebas unitarias simples y directas, depuración más sencilla y concurrencia simple.

Además, la adopción inevitable de CPUs con docenas de núcleos, y por lo tanto la creciente importancia de la programación no secuencial, solo acelerará el aumento en la adopción de la Programación Funcional, a medida que los viejos modelos de paralelismo traen complejidad que hace imposible razonar sobre los programas.

Con la realización de este trabajo se ha logró mostrar que un framework de juego programado en lenguaje \emph{Haskell} puede ser usado para la producción de videojuegos, el framework creado provee una manera transparente para aprovechar el uso de concurrencia en procesadores modernos y la capacidad de expresar la lógica del juego en forma funcional.

A lo largo de este trabajo se explica con mayor detalle las características que muestran los motores de juego y los lenguajes funcionales, la metodología usada en el desarrollo del framework, los detalles de implementación y uso del framework y finalmente se procederá a analizar el desempeño del mismos mediante programas de prueba. El \emph{Capítulo~\ref{capitulo2}} de este trabajo trata los temas teóricos de importancia para la comprención de este trabajo. El \emph{Capítulo~\ref{capitulo3}} se enfoca la metodología utilizada en la producción del programa. El \emph{Capítulo~\ref{capitulo4}} expone la implementación y características finales del programa. El \emph{Capítulo~\ref{capitulo5}} muestra los resultados obtenidos del uso de juegos implementados usando el programa creado. Finalmente el \emph{Capítulo~\ref{capitulo6}} expone las conclusiones sobre usar un lenguaje funcional para motores de juego.
