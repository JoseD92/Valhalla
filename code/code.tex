%!TEX root = ../thesis.tex

\chapter{Código}

\lstinputlisting[label={EasyGL},caption={EasyGL.hs},language=Haskell]{./code/src/EasyGL.hs}
\lstinputlisting[label={EasyGLUT},caption={EasyGLUT.hs},language=Haskell]{./code/src/EasyGLUT.hs}
\lstinputlisting[label={EasyGL.Camera},caption={Camera.hs},language=Haskell]{./code/src/EasyGL/Camera.hs}
\lstinputlisting[label={EasyGL.EasyMem},caption={EasyMem.hs},language=Haskell]{./code/src/EasyGL/EasyMem.hs}
\lstinputlisting[label={EasyGL.Entity},caption={Entity.hs},language=Haskell]{./code/src/EasyGL/Entity.hs}
\lstinputlisting[label={EasyGL.IndexedModel},caption={IndexedModel.hs},language=Haskell]{./code/src/EasyGL/IndexedModel.hs}
\lstinputlisting[label={EasyGL.Material},caption={Material.hs},language=Haskell]{./code/src/EasyGL/Material.hs}
\lstinputlisting[label={EasyGL.Obj},caption={Obj.hs},language=Haskell]{./code/src/EasyGL/Obj.hs}
\lstinputlisting[label={EasyGL.Obj.Grammar},caption={Grammar.y},language=Haskell]{./code/src/EasyGL/Obj/Grammar.y}
\lstinputlisting[label={EasyGL.Obj.Obj2IM},caption={Obj2IM.hs},language=Haskell]{./code/src/EasyGL/Obj/Obj2IM.hs}
\lstinputlisting[label={EasyGL.ObjData},caption={ObjData.hs},language=Haskell]{./code/src/EasyGL/Obj/ObjData.hs}
\lstinputlisting[label={EasyGL.Obj.Tokens},caption={Tokens.x},language=Haskell]{./code/src/EasyGL/Obj/Tokens.x}
\lstinputlisting[label={EasyGL.Shader},caption={Shader.hs},language=Haskell]{./code/src/EasyGL/Shader.hs}
\lstinputlisting[label={EasyGL.Texture},caption={Texture.hs},language=Haskell]{./code/src/EasyGL/Texture.hs}
\lstinputlisting[label={EasyGL.Util},caption={Util.hs},language=Haskell]{./code/src/EasyGL/Util.hs}

\lstinputlisting[label={Val.Strict},caption={EasyGL.hs},language=Haskell]{./code/src/Val/Strict.hs}
\lstinputlisting[label={Val.Strict.Data},caption={EasyGL.hs},language=Haskell]{./code/src/Val/Strict/Data.hs}
\lstinputlisting[label={Val.Strict.Events},caption={EasyGL.hs},language=Haskell]{./code/src/Val/Strict/Events.hs}
\lstinputlisting[label={Val.Strict.IL},caption={EasyGL.hs},language=Haskell]{./code/src/Val/Strict/IL.hs}
\lstinputlisting[label={Val.Strict.Scene},caption={EasyGL.hs},language=Haskell]{./code/src/Val/Strict/Scene.hs}
\lstinputlisting[label={Val.Strict.UI},caption={EasyGL.hs},language=Haskell]{./code/src/Val/Strict/UI.hs}
\lstinputlisting[label={Val.Strict.Util},caption={EasyGL.hs},language=Haskell]{./code/src/Val/Strict/Util.hs}
\lstinputlisting[label={Val.Strict.Scene.Resources},caption={EasyGL.hs},language=Haskell]{./code/src/Val/Strict/Scene/Resources.hs}

\begin{lstlisting}[label={Ejemplo1},caption={Ejemplo - Mostrando armadillo},language=Haskell]
import EasyGL
import EasyGLUT
import System.Exit
import System.IO (stderr)
import Control.Monad.IO.Class (MonadIO,liftIO)
import qualified Graphics.Rendering.OpenGL as GL

armadillo :: Shader -> IO (Material,Entity)
armadillo myShader = do
  m <- makeMaterial myShader []
  case m of
    Left s -> putStrLn s >> exitFailure
    Right mat -> do
      e <- readObj2Ent "./armadillo.obj"
      return (mat,e)

main = do
	-- se inicializa el contexto de OpenGL y GLUT.
  initOpenGLEnvironment 800 600 "test"
	-- se carga los shaders.
  myShader <-
		loadShadersFromFile
			["./vertex.shader","./frag.shader"]
			[VertexShader,FragmentShader]
			(Just stderr)
	-- se carga el mayado.
  assets <- armadillo myShader
  initGL $ myfun assets

-- funcion para el ciclo principal de la aplicacion.
myfun :: (Material,Entity) -> GLUT ()
myfun (mat,ent) = do
	liftIO $ GL.preservingMatrix $ do
		useCamera cam
		drawWithMat mat ent $
			set "color" $ GL.Color4 1 1 0 (1 :: GL.GLfloat)
	where
		(Right cam) =
			createCamera3D 0.0 0.0 10.0 0 0 0 30 (800/600) 0.3 200
\end{lstlisting}

\begin{lstlisting}[label={Ejemplo2},caption={Ejemplo - Objetos que chocan},language=Haskell]
{-# LANGUAGE Arrows #-}

import Val.Strict hiding (yaw)
import EasyGL
import EasyGLUT
import Data.Either
import System.IO (stderr)
import FRP.Yampa hiding (RandomGen,randomR)
import Data.List (find)
import System.Random
import System.Environment
import qualified Graphics.Rendering.OpenGL as GL
import qualified Data.Map as Map
import qualified System.Random.TF as TF

-- Se cargan los recursos a utilizar en el ejemplo
load :: IO ResourceMap
load = do
  myShader <- loadShadersFromFile
    ["./assets/3Dshaders/vertex.shader",
    "./assets/3Dshaders/ColorShader.shader"]
    [VertexShader,FragmentShader]
    (Just stderr)
  (Right mat) <- makeMaterial myShader []
  let plane = ("plane","./assets/plane.obj",mat)

  myShader <- loadShadersFromFile
    ["./assets/3Dshaders/vertex.shader",
    "./assets/3Dshaders/NormalShader.shader"]
    [VertexShader,FragmentShader]
    (Just stderr)
  (Right mat) <- makeMaterial myShader []
  let sphere = ("sphere","./assets/cube.obj",mat)

  loadResouces [plane,sphere]

-- Informacion de los objetos.
data GameState = Null
  | Sphere {
    x :: Double,
    y :: Double,
    z :: Double,
    size :: Double
  }

collition :: GameState -> GameState -> Bool
collition Sphere{x=x1,z=y1} Sphere{x=x2,z=y2} =
  ( (x1-x2)^2 + (y1-y2)^2 ) < 4
collition _ _ = False

data EventTypes = Collition GameState
  | NoCollition

instance MergeableEvent EventTypes where
  union NoCollition NoCollition = NoCollition
  union a NoCollition = a
  union NoCollition a = a
  union a b = a

-- Funcion que detecta colisiones.
collitionGen :: IL GameState -> IL (Event EventTypes)
collitionGen inObjs = mapILWithKey aux inObjs
  where
    assocs = assocsIL inObjs
    aux key obj = case valid of
      (x:_) -> Event $ Collition . snd $ x
      [] -> noEvent
      where
        valid = filter
          (\(key2,obj2) -> key /= key2 && collition obj obj2 )
          assocs

cam :: SF (GameInput,IL GameState) Camera3D
cam = proc _ -> do
    returnA -< c
    where
    (Right c) = createCamera3D 0 150 0 0 (-90) 0 30 (800/600) 0.3 200

plane :: Object GameState EventTypes
plane = proc _ -> do
  let ret = newObjOutput Null
      trans = Transform
        (GL.Vector3 0 (-1) 0)
        (Quaternion 0 (GL.Vector3 0 1 0))
        2 1 2
      uni = do
        set "color" $ GL.Color4 1 1 1 (1 :: GL.GLfloat)
  returnA -< ret{ooRenderer=Just("plane",trans,uni)}

moveSF :: GL.GLdouble
  -> GL.GLdouble
  -> (GL.GLdouble,GL.GLdouble)
  -> SF () (GL.GLdouble,GL.GLdouble)
moveSF limMax limMin d@(dir,_) =
  if (dir > 0) then aux2 (> limMax) (< limMin) d else aux2 (< limMin) (> limMax) d
  where
    aux2 f1 f2 (dir,initx) = switch (aux f1 dir initx) (aux2 f2 f1)
    aux f dir initx = proc _ -> do
      xnew <- (+initx) ^<< integral -< dir
      e <- edge -< f xnew
      returnA -< ((xnew,dir),tag e (-dir,xnew))

moveXZSF :: GL.GLdouble -> GL.GLdouble -> (GL.GLdouble,GL.GLdouble) ->
  GL.GLdouble -> GL.GLdouble -> (GL.GLdouble,GL.GLdouble) ->
  SF a (GL.GLdouble,GL.GLdouble,GL.GLdouble,GL.GLdouble)
moveXZSF limMaxX limMinX initX limMaxZ limMinZ initZ = proc _ -> do
  (x,dirx) <- moveSF limMaxX limMinX initX -< ()
  (z,dirz) <- moveSF limMaxZ limMinZ initZ -< ()
  returnA -< (x,z,dirx,dirz)

type Info = (GL.GLdouble,GL.GLdouble,GL.GLdouble,GL.GLdouble)

getCollition :: SF (ObjInput GameState EventTypes,Info) (Event Info)
getCollition = proc (oi,salida) -> do
  let myEvent = event noEvent toEvent $ inputEvent oi
  returnA -< tag myEvent salida
  where
    toEvent NoCollition = noEvent
    toEvent a = Event a

sphere :: GL.GLdouble -> GL.GLdouble
  -> GL.GLdouble -> GL.GLdouble
  -> Object GameState EventTypes
sphere initx initz velx velz = proc gi -> do
  rec
    rot <- impulseIntegral -< (180,tag e (-360))
    e <- iPre noEvent <<< edge -< rot > 360
    (x,z,dirx,dirz) <- moveArr (initx,initz,velx,velz) -< gi

  let ret = newObjOutput $ Sphere (realToFrac x) 0 (realToFrac z) 1
      trans = Transform
        (GL.Vector3 x 0 z)
        (Quaternion rot (GL.Vector3 0 1 0))
        1 1 1
  returnA -< ret{ooRenderer=Just("sphere",trans,return ())}
  where
    moveArr (x,z,velx,velz) = dkSwitch
      (moveXZSF 40 (-40) (velx,x) 40 (-40) (velz,z))
      (getCollition >>> notYet)
      (\sf (x,z,velx,velz) -> moveArr (x,z,-velx,-velz) )

randomSphere :: RandomGen g => g -> (Object GameState EventTypes, g)
randomSphere g = (sphere initx initz velx velz,g4)
  where
    (initx,g1) = randomR (-39,39) g
    (initz,g2) = randomR (-39,39) g1
    (velx,g3) = randomR (-4,4) g2
    (velz,g4) = randomR (-4,4) g3

randomSphereList :: RandomGen g => g -> [Object GameState EventTypes]
randomSphereList g = obj:(randomSphereList g1)
	where
		(obj,g1) = randomSphere g

main :: IO ()
main = do
  num <- fmap (read . head) getArgs
  gen <- TF.mkSeedTime >>= return . TF.seedTFGen
  let il = [plane] ++ (take num $ randomSphereList gen)
  initScenePar cam load [collitionGen] il
\end{lstlisting}
