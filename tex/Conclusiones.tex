\documentclass{standalone}
\begin{document}

\paragraph{}
La programación funcional permite la creación de juegos que aprovechen la concurrencia provista por los procesadores modernos sin la necesidad de introducir una mayor complejidad para lidiar con los problemas de la programación paralela. Vemos en el documento de Jeff Andrews \cite{andrews2009designing} la clase de estructuras y sistemas usados en los lenguajes convencionales para poder hacer uso del paralelismo, pero la inmutabilidad de los datos en haskell resuelve todos los problemas para paralelismo en los juegos.

\paragraph{}
Adicionalmente a ello, sistema de tipos de haskell permite razonar más sobre la lógica del juego y a la detección temprana de errores que en lenguajes convencionales no podrían ser notados. La implementación sencilla de la concurrencia de este motor también ayuda a reducir todos los errores causados en los juegos por llevar los sistemas a un entorno concurrente.

\end{document}
