% ************************** Thesis Abstract *****************************
% Use `abstract' as an option in the document class to print only the titlepage and the abstract.
\begin{abstract}

Los avances recientes en la informática han visto a los lenguajes funcionales conducir a una mejor productividad en muchas industrias, y la industria de los videojuegos y medios interactivos también se pueda beneficiar de estos avances con el uso de lenguajes funcionales. Los lenguajes de programación funcionales ofrecen muchas ventajas comparadas con los lenguajes imperativos que son ampliamente utilizados en esta industria. Los lenguajes funcionales son mucho más concisos en comparación con los lenguajes imperativos, y está probado, que en ciertas ocasiones, los lenguajes funcionales muestran un mejor rendimiento que sus contrapartes imperativas. También permiten el uso de poderosas abstracciones que pueden ser utilizadas para mejorar la estructura y modularidad del código. Los lenguajes funcionales también permiten el polimorfismo promoviendo la reutilización del código y una menor redundancia en los programas.

Es por estas razones que este proyecto se propone la creación de un framework que permita la construcción de juegos en el lenguaje \emph{Haskell}, a través de una interfaz que permita, de manera transparente al programador, implementar la funcionalidad requerida para su propio proyecto sin tener que preocuparse por la lógica y funcionalidad común en todo videojuego, que en el lenguaje \emph{Haskell}, el cual presenta sus propios retos y complicaciones diferentes a los encontrados en lenguajes imperativos tradicionales como C.

\paragraph{Palabras clave}
Programación funcional. Haskell. Motores de juegos. Videojuegos. FRP. 

\end{abstract}
