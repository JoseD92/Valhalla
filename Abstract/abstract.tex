% ************************** Thesis Abstract *****************************
% Use `abstract' as an option in the document class to print only the titlepage and the abstract.
\begin{abstract}

La industria moderna de videojuegos maneja algunos de los proyectos más grandes y complejos llevados a cabos en el área de la informática, y con los riesgos financieros que estos proyectos implican, llevan a esta industria a la búsqueda de nuevas formas de aumentar la productividad, reducir errores y adaptar nuevas y cambiantes tecnologías a los ciclos de desarrollo de juegos. Sin embargo, en la actualidad esta industria sufre de ciclos de desarrollo cortos con equipos que comprendes miles de personas, llevando a código que muchas veces contiene fallas o es lo suficientemente complicado para hacer imposible su crecimiento y expansión.

El presente trabajo propone una alternativa para la producción de videojuegos en la forma de un motor de juego que permita la producción de los mismos en un lenguaje funcional, en este caso el lenguaje Haskell, para así poder hacer uso de las ventajas que este tipo de lenguajes provee con el fin de mejorar la productividad y calidad del producto.

El prototipo creado en el desarrollo de este trabajo usa el paradigma de programación funcional reactiva como substituto del sistema entidad-componente popular en la industria de videojuegos. Este paradigma permite a los objetos del juego ser representados como cómputos puros en el contexto de Haskell, lo que permite al motor y al compilador realizar mejoras en la corrida del juego.

El motor también permite, en forma transparente al usuario, ejecutar código en forma concurrente llevando a una mejora en el rendimiento de los juegos creados y así a la experiencia de juego.

En resumidas cuentas, un motor de juego puede beneficiarse de usar un lenguaje funcional como base y permitir al programador enfocarse en la jugabilidad mientras que el motor garantiza una mayor correctitud del código y a la vez que hace un mayor uso de los recursos del sistema.

Futuras mejoras de este motor pueden enfocarse en agregar funcionalidades en la forma de subsistemas, como un motor de física o de inteligencia artificial, o en mejorar la usabilidad.

\end{abstract}
