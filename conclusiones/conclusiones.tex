%!TEX root = ../thesis.tex

\chapter{Conclusiones}
\label{capitulo6}

Con la realización de este trabajo se logró la creación de una herramienta que simplifica la creación de juegos en un entorno de programación funcional. La herramienta creada permite el despliegue grafico de objetos, el manejo de la entrada del usuario, pedidos de E/S y la actualización, de forma pura, entidades de juego que interactúan entre sí que conforman la escena del juego.

Un avance importante que provee la herramienta creada al conocimiento de motores de juego está en la habilidad de optimizar para el paralelismo. Como la herramienta requiere que el usuario programe las entidades del juego en forma pura, es posible ejecutar el código del usuario en un orden arbitrario sabiendo que este no interactúa con otros elementos y que su resultado no será alterado. Esta flexibilidad permite procesar la actualización de cada entidad en hilos de cómputo separados. Adicionalmente, como la información de las entidades es inmutable, se puede compartir con otros hilos de cómputo que usen la información para operaciones de entrada y salida (como graficar) de forma segura.

Esta misma pureza en la programación de las entidades, combinada con un sistema de tipos estricto como el de \emph{Haskell}, permite que la herramienta sea útil para detectar errores de forma temprana en el código de nuestros juegos.

La estructura del motor permite que nueva funcionalidad  sea fácilmente añadida, solo se tiene que pedir a las entidades que su salida provea la interfaz requerida por la nueva funcionalidad. Cualquier nueva funcionalidad, como por ejemplo un motor de física, podría fácilmente ser añadido a la herramienta para que corra en un hilo de computo independiente, ya que esta solo requeriría un apuntador a la información de las entidades y esta información es inmutable, permitiendo al motor de física realizar su función sin tener que cambiar otros sistemas para poder añadirlo.

La experiencia con el motor muestra la conveniencia de usar FRP como máquinas de estado, comparando con una implementación imperativa, como la propuesta en el libro “Artificial intelligence for games” \cite{millington2016artificial}, FRP resulta mucho más modular y re utilizable mientras que esconde la transiciones entre los estados.

También es de notar que durante la creación de juego de prueba hechos usando la herramienta, se realizaron múltiples cambios en la interfaz que permite la interacción de las entidades del juego con los diferentes sistemas de la herramienta, esto es debido a que en ciertas circunstancias ciertas formas de hacer las cosas resulta más cómodo, y con una herramienta tan joven como esta, solo su uso continuo proveerá una mejor visión de los cambios requeridos para hacer su uso más placentero a cualquier posible usuario en un futuro, esta es una herramienta que todavía posee espacio para mejorar y ser expandida.
